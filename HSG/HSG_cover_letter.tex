\documentclass[11pt]{article}
\usepackage[margin=1in]{geometry}
\usepackage{parskip}
\usepackage{hyperref}
\usepackage{lmodern}
\usepackage[T1]{fontenc}
\usepackage{setspace}

\hypersetup{
  colorlinks=true,
  linkcolor=black,
  urlcolor=blue
}

\begin{document}

\noindent\textbf{\Large Amirehsan Davoodi}

\vspace{6pt}

\noindent\begin{tabular*}{\textwidth}{@{\extracolsep{\fill}} l r}
    \href{mailto:amirehsan.davoodi@gmail.com}{amirehsan.davoodi@gmail.com} & Tehran, Iran \\
    LinkedIn: \href{https://www.linkedin.com/in/amirehsan-davoodi}{linkedin.com/in/amirehsan-davoodi} & (+98)~915\,612\,6388 \\
\end{tabular*}

\vspace{6pt}

\noindent\begin{flushright}
    \today
\end{flushright}

\vspace{0.5cm}

\textbf{Prof. Dr. Ivo Blohm}\\
Institute of Information Systems and Digital Business (IWI--HSG)\\
University of St. Gallen\\
St. Gallen, Switzerland\\

\vspace{0.5cm}
\textbf{Subject: Application for PhD candidate / Research Associate -- Agentic AI and Human--AI Collaboration (Job ID 1557)}

Dear Prof. Dr. Blohm and Hiring Committee,

I am writing to apply for the PhD candidate / Research Associate position in Agentic AI and Human--AI Collaboration (Job ID 1557) at IWI--HSG. I recently saw your LinkedIn post and appreciate that you accepted my connection request. The role aligns closely with my current doctoral research and my industry experience building AI-enabled software systems. The official position description is available at \href{https://jobs.unisg.ch/offene-stellen/phd-candidate-research-associate-in-the-field-of-agentic-ai-and-human-ai-collaboration-m-f-d/fa3ee000-108b-4124-8af3-63ade955e061}{jobs.unisg.ch}.

I am a PhD candidate in Artificial Intelligence at Amirkabir University of Technology (AGML Center), and I hold an MSc in Artificial Intelligence from USI (Universit\`a della Svizzera italiana), Lugano. My Master's thesis, ``Goal-directed graph generation for anomaly detection on time series,'' investigated graph-based representations for time-series anomaly detection (case study: ECG arrhythmia). Recent work includes ASD-GraphNet, a graph learning approach for fMRI-based autism diagnosis (Computers in Biology and Medicine, 2025). During my Master's, I co-founded the startup UbiHealth with fellow USI students; our work on remote patient monitoring was covered by USI (\href{https://www.usi.ch/en/feeds/8176}{USI News}). I also successfully completed the Innosuisse Entrepreneurship Training in Ticino, Switzerland, which strengthened my skills in translating research prototypes into viable products.

In parallel, I have a paper under review on an agentic AI approach to knowledge graph construction: HIDE-KG (Hierarchical Dual-learning Entity-clustered Knowledge Graph Construction Using Pre-trained LLMs). The core contribution is a dual-learning architecture that validates the constructed knowledge graph via text-based reconstruction and re-parsing, reducing hallucinations and improving reliability. I see knowledge graphs as an effective representation to extract and organize hidden knowledge across multi-source enterprise data streams, which directly connects to organizational AI adoption and the design of practical agentic assistants.

Beyond research, I bring several years of hands-on software engineering experience (Full-stack/Backend; FastAPI, Node.js, PostgreSQL/MongoDB, Docker). At Tali AI (Toronto), I contributed to production-grade AI assistants and data platforms. This background equips me to deliver the position's emphasis on methods, prototypes, and experiments---from Python-based ML/DL development (PyTorch, scikit-learn, XGBoost) and LLM/agent frameworks to robust API services (FastAPI) and containerized deployments.

I am motivated to study how agentic AI systems can augment human work in organizations---designing and experimentally evaluating chat/voice agents and multi-agent workflows that improve creativity, productivity, and well-being. I am particularly interested in: (1) systematic design and implementation of reliable agentic systems with tool-use, planning, and memory; (2) rigorous lab and field experiments (including A/B tests) to quantify outcomes and boundary conditions (task complexity, trust, oversight); and (3) governance and alignment practices for organizational adoption. These interests align with your group's focus on integrating (Generative) AI in organizational processes and evaluating impact in real settings.

My profile matches the advertised criteria: strong foundation in ML/DL/NLP and Generative AI; applied prototyping of LLM-powered assistants; solid Python engineering with experience in FastAPI and containerization; and a publication-oriented mindset. I am fluent in English and am improving my German. Given that my current supervisor has relocated abroad due to regional instability, I am seeking a stable, research-active environment to complete a timely (\~3.5 years) and impactful PhD. Your team at IWI--HSG is an ideal fit.

I would be grateful for your support and consideration of my application. I would welcome the opportunity to discuss how my background can contribute to your research agenda in Agentic AI and Human--AI Collaboration. Thank you for your time.

Sincerely,\\[6pt]
\textbf{Amirehsan Davoodi}

\end{document}


